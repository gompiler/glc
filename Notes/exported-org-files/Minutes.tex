% Created 2019-02-11 Mon 15:28
% Intended LaTeX compiler: pdflatex
\documentclass[11pt]{article}
\usepackage[utf8]{inputenc}
\usepackage[T1]{fontenc}
\usepackage{graphicx}
\usepackage{grffile}
\usepackage{longtable}
\usepackage{wrapfig}
\usepackage{rotating}
\usepackage[normalem]{ulem}
\usepackage{amsmath}
\usepackage{textcomp}
\usepackage{amssymb}
\usepackage{capt-of}
\usepackage{hyperref}
\date{\today}
\title{Meeting Minutes and Discussion Notes}
\hypersetup{
 pdfauthor={},
 pdftitle={Meeting Minutes and Discussion Notes},
 pdfkeywords={},
 pdfsubject={},
 pdfcreator={Emacs 26.1 (Org mode 9.2.1)}, 
 pdflang={English}}
\begin{document}

\maketitle
\tableofcontents

This document is for jotting down quick notes of our meetings an
discussions and is not the design document (although many parts of
here will formally be included in said document).
\section{\textit{<2019-02-11 Mon>}}
\label{sec:org784dcfd}
\subsection{Comparison of Allan and Julian's Methods of Writing a Compiler in Haskell}
\label{sec:org09a3bac}
\subsubsection{Compared Parsing Methods (Megaparsec vs Alex \(\to\) Happy)}
\label{sec:org5c27e6b}
\begin{itemize}
\item Alex/Happy is much more similar to Flex/Bison
\begin{itemize}
\item Will be easier for David
\end{itemize}
\item Could generate tokens with Alex and feed them to Megaparsec
\begin{itemize}
\item This would be more difficult as we'd have to learn how to use
Megaparsec in new ways (i.e. with a token stream)
\end{itemize}
\item Megaparsec errors are much nicer
\begin{itemize}
\item However, they only require the input string and the offset, all of
which can be obtained from Alex if using the linepos or monadic wrapper
\end{itemize}
\item Simplest approach would be to use Alex/Happy
\begin{itemize}
\item Mainly have David and Julian work on scanning \& parsing since they
are familiar with the toolsets/style
\end{itemize}
\end{itemize}
\subsubsection{Discussion of Project Name}
\label{sec:org0bc3839}
\begin{itemize}
\item Settled on \texttt{glc} as package/executable name
\item Project name is either Gophiler, Gompiler or Gopiler
\begin{itemize}
\item Settled on Gompiler
\end{itemize}
\end{itemize}
\end{document}