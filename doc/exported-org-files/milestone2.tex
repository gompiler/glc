% Created 2019-03-17 Sun 15:02
% Intended LaTeX compiler: pdflatex
\documentclass[11pt]{article}
\usepackage[utf8]{inputenc}
\usepackage[T1]{fontenc}
\usepackage{graphicx}
\usepackage{grffile}
\usepackage{longtable}
\usepackage{wrapfig}
\usepackage{rotating}
\usepackage[normalem]{ulem}
\usepackage{amsmath}
\usepackage{textcomp}
\usepackage{amssymb}
\usepackage{capt-of}
\usepackage{hyperref}
\usepackage[margin=0.9in]{geometry}
\usepackage[fontsize=10.5pt]{scrextend}
\author{Lore, J., Lougheed D., Wang A.}
\date{\today}
\title{Design Document for Milestone 1}
\hypersetup{
 pdfauthor={Lore, J., Lougheed D., Wang A.},
 pdftitle={Design Document for Milestone 1},
 pdfkeywords={},
 pdfsubject={},
 pdfcreator={Emacs 26.1 (Org mode 9.2.2)}, 
 pdflang={English}}
\begin{document}

\maketitle
\tableofcontents

This document is for explaining the design decisions we had to make
whilst implementing the components for milestone 2.  \newpage
\section{Design Decisions}
\label{sec:org7dd8775}
\subsection{Weeding}
\label{sec:org48d38f9}
We implemented additional weeding passes for certain constraints
that could be verified either at the weeding level or the typecheck
level because it was easier to check via weeding. The constraints
we checked in weeding for this milestone are:
\begin{itemize}
\item Correct use of the blank identifier. It was much easier to recurse
through the AST and gather all identifiers that cannot be blank
and then just check this whole list. Additionally, because we
have offsets in our AST, we could easily point to the offending
blank identifier without having to make error messages for each
specific incorrect usage, as it is obvious what the incorrect
usage when we print out where the blank identifier is.
\item Function bodies ending in return statements. It was easier to do
a single weeding pass because otherwise we'd have to recurse
differently on functions that have a return type versus functions
that don't have a return type during typechecking, essentially
we'd need two versions of typechecking statements, which was not
deemed worth it compared to a single weeding pass.
\item \texttt{init} function declaration not having any non void return
statements. Similar argument to the above, it is easier to do a
weeding pass then require context/a different traversal function
given the context of our current function.
\item `main` function cannot return non void. Straightforward weeding
check. Could also have been done via typecheck, but it is
slightly easier to implement weeding passes (no symbol table
required).
\end{itemize}
\subsection{Symbol Table}
\label{sec:org50fb42e}
In Haskell, data structures are typically immutable, and much of
the language is designed around this. One of the main design
decisions made around the symbol table was deciding whether to go
with an immutable or mutable symbol table. In an immutable symbol
table, a new symbol table would have to be made every time a scope
is added or modified. Right away, despite this being a conceptually
better fit for the language, the potential performance degradation
of constantly re-building the symbol table becomes evident.

As a result of this performance impact, we decided on using a
mutable symbol table, with mutability supported via Haskell's
\href{https://hackage.haskell.org/package/base-4.12.0.0/docs/Control-Monad-ST.html}{ST}
monad. The constraint provided for
\href{https://hackage.haskell.org/package/base-4.12.0.0/docs/Control-Monad-ST.html\#v:runST}{runST}
(which removes the \texttt{ST} (or mutability) from something) is proven
to keep functions pure (see
\href{https://iris-project.org/pdfs/2018-popl-runST-final.pdf}{A
Logical Relation for Monadic Encapsulation of State by Amin Timany
et al.}). This made the \texttt{ST} monad a better choice than other
monads providing mutability (the main example being \texttt{IO}, but if we
used \texttt{IO} then all our functions after the symbol table generation
would be bound by \texttt{IO}, i.e. impure and also harder to deal with as
they are wrapped by an unnecessary monad). The trade-off of this
decision was a large increase in the difficulty implementing the
symbol table, which made up a huge portion of the work for this
milestone, however, once we finish using the symbol table, our
final result (a typechecked/simplified proven correct AST) is pure
and very easy to manipulate for \texttt{codegen} in the next milestone.
\subsubsection{Scoping Rules}
\label{sec:orged97ce6}
The scoping rules we used/considered are as follows:
\begin{itemize}
\item Function declarations: the parameters and function body are put
in a new scope, but the function itself is declared at the
current scope. Note that here we had to treat the function body
as a list of statements and not a block statement, because if we
recursed on the block statement our block statement rule would
put the function body in a new scope, but it has to be in the
same scope as the parameters.
\item Block statements are put into a new scope
\item If statements: we open a new scope (containing the simple
statement and expression condition at the top level) and then
another scope inside for the body/bodies (one for if, one for
else if there's one, in that case the if and else scope are
siblings).
\item Switch statements: open a new scope and another scope for each
switch case (all switch cases have sibling scopes to each other)
\item For loop: open a new scope, optional clauses are put at the top
level (simple statement 1 and 2 and condition) and the body is
put in a nested scope
\end{itemize}
\subsection{Type Checker}
\label{sec:orgadbcb1f}
For type-checking, we decided on a single-pass approach which
combined symbol table generation and statement type-checking. This
improves performance, and is possible as a product of GoLite's
declaration rules, which specify that identifiers must be declared
before they can be used. The other approach we considered was
having a type annotated AST (types of expressions would be
contained in the AST) so that we could get rid of mutability (the
symbol table) as soon as possible (also some of us did a similar
thing for the assignment, but this was mainly relevant for the fact
that print statements in codegen need to know the type of the
expression they're printing in C), however we decided on doing all
the typechecking at the same time as symbol table generation
because type inference has to be done to generate this new AST and
type inference requires typechecking (\texttt{"a" + 5} has no inferred
type, but we only know that because we typecheck it). Therefore
we'd generate an annotated AST only to typecheck things that aren't
expressions. But at that point, since we are already doing one
in-depth pass of the original AST when generating the symbol table,
we might as well do the other half of typechecking at the same
phase (it seemed weird to split half of typechecking with a symbol
table and half without it and might have been more feasible if type
inference did not require typechecking, but that makes no
sense). Therefore, after the one pass of our original AST, the
final result is a typechecked AST with no type annotation.

Additionally, we decided to resolve all type mappings (except for
structs) to their base types when generating this new AST (all the
casts/equality checks/new type usages are already validated in
typechecking, so we don't need them anymore, nor do we need the
mappings). Therefore our new AST was also able to get rid of type
declarations (except for structs).
\subsection{Invalid Programs}
\label{sec:org8c0b191}
Summary of the check in each invalid program:
\begin{itemize}
\item \texttt{append-diff-type.go}: Append an expression of a different type than
the type of the expressions of the \texttt{slice}.
\item \texttt{append-no-slice.go}: Append to something that isn't a slice.
\item \texttt{assign-no-decl.go}: Assign to a variable that hasn't been declared.
\item \texttt{assign-non-addressable.go}: Assign to a LHS that is a
non-addressable field.
\item \texttt{cast-not-base.go}: Cast to a type that isn't a base type.
\item \texttt{dec-non-lval.go}: Decrement something that isn't an \texttt{lvalue}.
\item \texttt{decl-type-mismatch.go}: Declare and assign variable of explicit type
to an expression of a different type.
\item \texttt{float-to-string.go}: Try to cast a \texttt{float} to a \texttt{string}.
\item \texttt{for-no-bool.go}: While variant of for loop with a condition that isn't
a bool.
\item \texttt{func-call.go}: Function call with arguments of different type than
function declaration arguments.
\item \texttt{func-no-decl.go}: Calling a function that hasn't been declared.
\item \texttt{function-already-declared.go}: Trying to declare a function that
has already been declared.
\item \texttt{function-duplicate-param.go}: Trying to declare function with two
params with same name.
\item \texttt{if-bad-init.go}: If with an init statement that does not typecheck
(assignment of different type).
\item \texttt{inc-non-numeric.go}: Increment an expression that doesn't resolve
to a numeric base type.
\item \texttt{index-not-list.go}: Index into something that isn't a slice.
\item \texttt{index.go}: Index that does not resolve to an int.
\item \texttt{invalid-type-decl.go}: Declare a type mapping to a type that
doesn't exist.
\item \texttt{no-field.go}: Using selector operator on struct that doesn't have
the field requested.
\item \texttt{non-existent-assign.go}: Assigning a variable to a non existent
variable.
\item \texttt{non-existent-decl.go}: Trying to declare a variable of a type that
doesn't exist.
\item \texttt{op-assign.go}: Op-assignment where variable and expression are not
compatible with operator (i.e. \texttt{int + string})
\item \texttt{print-non-base.go}: Trying to print a non base type.
\item \texttt{return-expr.go}: Returning an expression of different type than the
return type of the function.
\item \texttt{return.go}: Return nothing from non-void function.
\item \texttt{short-decl-all-decl.go}: Short declaration where all variables on
LHS are already declared.
\item \texttt{short-decl-diff-type.go}: Short declaration where already defined
variables on LHS are not the same type as assigned expression.
\item \texttt{switch-diff-type.go}: Type of expression of case is different from
switch expression type.
\item \texttt{type-already-declared.go}: Trying to define a type mapping to a
type that already exists.
\item \texttt{var-already-declared.go}: Trying to declare a variable that is
already declared.
\end{itemize}
\section{Team}
\label{sec:org3d6a272}
\subsection{{\bfseries\sffamily TODO} Team Organization}
\label{sec:org8b263a8}
\subsection{Contributions}
\label{sec:org0725b0c}
\begin{itemize}
\item \textbf{Julian Lore:} Implemented weeding of blank identifiers, symbol
table generation, typecheck (aside from type inference and
expression typechecking)
\item \textbf{David Lougheed:} Worked on expression type-checking and type inference,
including tests. Also worked on the weeding pass for return
statements.
\item \textbf{Allan Wang:} TODO
\end{itemize}
\end{document}