% Created 2019-03-17 Sun 23:50
% Intended LaTeX compiler: pdflatex
\documentclass[11pt]{article}
\usepackage[utf8]{inputenc}
\usepackage[T1]{fontenc}
\usepackage{graphicx}
\usepackage{grffile}
\usepackage{longtable}
\usepackage{wrapfig}
\usepackage{rotating}
\usepackage[normalem]{ulem}
\usepackage{amsmath}
\usepackage{textcomp}
\usepackage{amssymb}
\usepackage{capt-of}
\usepackage{hyperref}
\usepackage[margin=0.9in]{geometry}
\usepackage[fontsize=10.5pt]{scrextend}
\author{Lore, J., Lougheed D., Wang A.}
\date{\today}
\title{Design Document for Milestone 2}
\hypersetup{
 pdfauthor={Lore, J., Lougheed D., Wang A.},
 pdftitle={Design Document for Milestone 2},
 pdfkeywords={},
 pdfsubject={},
 pdfcreator={Emacs 26.1 (Org mode 9.2.2)}, 
 pdflang={English}}
\begin{document}

\maketitle
\tableofcontents

This document is for explaining the design decisions we had to make
whilst implementing the components for milestone 2.  \newpage
\section{Weeding}
\label{sec:orgea6003d}
We implemented additional weeding passes for certain constraints
that could be verified either at the weeding level or the typecheck
level, since in most cases it was easier to check via weeding. The
constraints we use weeding to check for this milestone are:
\begin{itemize}
\item Checking correct use of the blank identifier. It was much easier to recurse
through the AST and gather all identifiers that cannot be blank,
and then check this whole list, versus checking usage in the
type-checking pass. Additionally, because we have offsets in our
AST, we could easily point to the offending blank identifier
without having to make error messages for each specific incorrect
usage, as it is obvious what the incorrect usage is when we print
out the location of the blank identifier.
\item Ensuring non-void function bodies end in return statements. It was easier to
do a single weeding pass; otherwise, we'd have to recurse
differently on functions that have a return type versus functions
that don't have a return type during typechecking. Essentially,
we'd need context-sensitive statement typechecking, with two
different versions depending on return type. This was not deemed
worth it compared to a single weeding pass.
\item Ensuring \texttt{init} function declarations do not have any non-void return
statements. Similar to the above, it is easier to do a weeding
pass then require context-sensitive (i.e multiple different)
traversal functions given the context of the current function
declaration.
\item Ensuring any `main` or `init` function declarations do not have any
parameters or a return type. This is again a straightforward
weeding check.  It could also have been done via typecheck, but it
is slightly easier to implement with weeding passes (no symbol
table required).
\item Checking that any top-level declarations with the identifier \texttt{init} or
\texttt{main} are functions, since they are special identifiers in Go
(and GoLite) in the global scope.
\end{itemize}
\section{Symbol Table}
\label{sec:org949e50d}
In Haskell, data structures are typically immutable, and much of the
language is designed around this. One of the main design decisions
made around the symbol table was deciding whether to go with an
immutable or mutable symbol table. In an immutable symbol table, a
new symbol table would have to be made every time a scope is added
or modified. Right away, despite this being a conceptually better
fit for the language, the potential performance degradation of
constantly re-building the symbol table becomes evident.

As a result of this performance impact, we decided on using a
mutable symbol table, with mutability supported via Haskell's
\href{https://hackage.haskell.org/package/base-4.12.0.0/docs/Control-Monad-ST.html}{ST}
monad. The constraint provides
\href{https://hackage.haskell.org/package/base-4.12.0.0/docs/Control-Monad-ST.html\#v:runST}{runST}
(which removes the \texttt{ST}, i.e.  mutability, from an interior data
structure). This is proven to keep functions pure (see
\href{https://iris-project.org/pdfs/2018-popl-runST-final.pdf}{A Logical
Relation for Monadic Encapsulation of State by Amin Timany et
al.}). This made the \texttt{ST} monad a better choice than other monads
providing mutability (the main example being \texttt{IO}, which if used
would have resulted in all our functions after symbol table
generation being bound by \texttt{IO}, i.e. impure and also harder to work
with, as they are wrapped by an unnecessary monad). The trade-off of
this decision was a large increase in the difficulty of implementing
the symbol table, which made up a huge portion of the work for this
milestone.  However, once we finish using the symbol table, our
final result (a typechecked/simplified, proven-correct AST) is pure
and very easy to manipulate for the \texttt{codegen} phase in the next
milestone.

For symbol table storage, we created a new data type \texttt{Symbol} to
represent each symbol, analogous to the different type of symbols
(types, constants, functions, and variables). A \texttt{Symbol} can be one
of:
\begin{itemize}
\item \texttt{Base}: Base types (\texttt{int}, \texttt{float64}, etc.)
\item \texttt{Constant}: Constant values (only used for booleans in GoLite)
\item \texttt{Func [Param] (Maybe SType)}: Function types, with parameters and an
optional return type.
\item \texttt{Variable SType}: Declared variables, of type SType.
\item \texttt{SType SType}: Declared types (note: in Haskell, the first \texttt{SType} here is
a constructor, and the second \texttt{SType} is a data type attached to
the constructor.)  In this way, all possible symbol types are
encompassed by a single Haskell data type.
\end{itemize}

We also created a new data type \texttt{SType}, which is used to store
vital information concerning the types we define in the original
\texttt{AST}.  An \texttt{SType} can be one of:
\begin{itemize}
\item \texttt{Array Int SType}: An array with a length and a type.
\item \texttt{Slice SType}: A slice with a type.
\item \texttt{Struct [Field]}: A struct with a list of fields.
\item \texttt{TypeMap SIdent SType}: A user-defined type, with an identifier and an
underlying type, which may be recursively mapped, eventually to a
base type.
\item \texttt{PInt, PFloat64, PBool, PRune, PString}: Base types.
\item \texttt{Infer, Void}: Special \textasciitilde{}SType\textasciitilde{}s for inferred types and void return values.
\end{itemize}
In this way, all possible GoLite types, including user-defined
types, are accounted for.

All types that are left to be inferred during symbol table
generation are updated when typechecking (we infer the type of
variables and then update their value in the symbol table, to make
sure things like assignments don't conflict with the original
inferred type).
\subsection{Scoping Rules}
\label{sec:org6db7fbb}
The scoping rules we used/considered are as follows:
\begin{itemize}
\item Function declarations: the parameters and function body are put
in a new scope, but the function itself is declared at the
current scope. Note that here we had to treat the function body
as a list of statements and not a block statement, because if we
recursed on the block statement our block statement rule would
put the function body in a new scope. Instead, the body must be
in the same scope as the parameters.
\item Block statements are put into a new scope.
\item If statements: we open a new scope, containing the simple
statement and expression condition at the top level, and then
other scope(s) inside for the body/bodies: one for \texttt{if}, and one
for \texttt{else} (if there is one). If an \texttt{else} is present, the if and
else scope are siblings.
\item Switch statements: open a new scope for the \texttt{switch}, and another scope
for each switch \texttt{case}. All switch case scopes are siblings.
\item For loop: open a new scope, with optional clauses put at the top
level (simple statements 1 and 2, and condition). The body is put
in a nested scope.
\end{itemize}
\section{Type Checker}
\label{sec:org5ddbc2e}
For type-checking, we decided on a single-pass approach, combining
combined symbol table generation and statement type-checking. This
improves performance, and is feasible as a product of GoLite's
declaration rules, which specify that identifiers must be declared
before they can be used.

The other approach we considered involved generation of a
type-annotated AST, with types of expressions contained in the AST,
so that we could get rid of \texttt{ST} mutability from the symbol table as
soon as possible (some of us did a similar thing for the assignment,
but this was mainly relevant for print statements in C codegen
needing to know the type of the expression they're printing).

We decided on doing all typechecking at the same time as symbol
table generation because type inference has to be done to generate
this new AST, and type inference requires typechecking (e.g. \texttt{"a" +
  5} has no inferred type, since the expression is undefined; we only
know this because of typechecking).

At first, we were going to generate an annotated AST only to
typecheck things that aren't expressions. However, at that point,
since we were already doing one in-depth pass of the original AST
for symbol table generation, we decided that we might as well do the
other half of typechecking in the same phase, since it seemed odd to
split typechecking between the symbol table and a separate pass. The
alternate approach may have been more feasible if type inference did
not require typechecking, but in GoLite it did not seem to make
sense. Therefore, after the one pass of our original AST, the final
result is a typechecked AST, with extremely limited type annotation.

Additionally, we decided to resolve all type mappings (except for
structs) to their base types when generating this new AST: all the
casts/equality checks/new type usages are already validated in
typechecking, so we don't need them anymore, nor do we need the
mappings. Thus, our new AST was also able to get rid of type
declarations (except for structs).
\section{New AST}
\label{sec:org3c20e48}
As mentioned above, dependency on the SymbolTable results in a
dependency on the \texttt{ST} monad, which adds complexity to each
operation.  As a result, our goal after typechecking is to create a
new AST, which reflects the new constraints we enforce.  Namely:
\begin{itemize}
\item Typecheck errors are caught beforehand, so we no longer need offsets,
or error breakpoints.
\item All variables are properly typechecked, and can therefore reference an
explicit type. Each type is composed of parent types up until the
primitives.  This includes cases like function signatures, where
we can associate each parameter with a type instead of allowing
lists of identifiers to map to a single type.  In preparation for
codegen, we can then use our new AST exclusively, without any
other mutable data structures. Any additinoal information we need
can be added back into the AST, with minimal changes to models
used at previous stages.
\end{itemize}

\section{Invalid Programs}
\label{sec:org2b7c392}
Summary of the check in each invalid program:
\begin{itemize}
\item \texttt{append-diff-type.go}: Append an expression of a different type than
the type of the expressions of the \texttt{slice}.
\item \texttt{append-no-slice.go}: Append to something that isn't a slice.
\item \texttt{assign-no-decl.go}: Assign to a variable that hasn't been declared.
\item \texttt{assign-non-addressable.go}: Assign to a LHS that is a
non-addressable field.
\item \texttt{cast-not-base.go}: Cast to a type that isn't a base type.
\item \texttt{dec-non-lval.go}: Decrement something that isn't an \texttt{lvalue}.
\item \texttt{decl-type-mismatch.go}: Declare and assign variable of explicit type
to an expression of a different type.
\item \texttt{float-to-string.go}: Try to cast a \texttt{float} to a \texttt{string}.
\item \texttt{for-no-bool.go}: While variant of for loop with a condition that isn't
a bool.
\item \texttt{func-call.go}: Function call with arguments of different type than
function declaration arguments.
\item \texttt{func-no-decl.go}: Calling a function that hasn't been declared.
\item \texttt{function-already-declared.go}: Trying to declare a function that
has already been declared.
\item \texttt{function-duplicate-param.go}: Trying to declare function with two
params with same name.
\item \texttt{if-bad-init.go}: If with an init statement that does not typecheck
(assignment of different type).
\item \texttt{inc-non-numeric.go}: Increment an expression that doesn't resolve
to a numeric base type.
\item \texttt{index-not-list.go}: Index into something that isn't a slice.
\item \texttt{index.go}: Index that does not resolve to an int.
\item \texttt{invalid-type-decl.go}: Declare a type mapping to a type that
doesn't exist.
\item \texttt{no-field.go}: Using selector operator on struct that doesn't have
the field requested.
\item \texttt{non-existent-assign.go}: Assigning a variable to a non existent
variable.
\item \texttt{non-existent-decl.go}: Trying to declare a variable of a type that
doesn't exist.
\item \texttt{op-assign.go}: Op-assignment where variable and expression are not
compatible with operator (i.e. \texttt{int + string})
\item \texttt{print-non-base.go}: Trying to print a non base type.
\item \texttt{return-expr.go}: Returning an expression of different type than the
return type of the function.
\item \texttt{return.go}: Return nothing from non-void function.
\item \texttt{short-decl-all-decl.go}: Short declaration where all variables on
LHS are already declared.
\item \texttt{short-decl-diff-type.go}: Short declaration where already defined
variables on LHS are not the same type as assigned expression.
\item \texttt{switch-diff-type.go}: Type of expression of case is different from
switch expression type.
\item \texttt{type-already-declared.go}: Trying to define a type mapping to a
type that already exists.
\item \texttt{var-already-declared.go}: Trying to declare a variable that is
already declared.
\end{itemize}
\section{Team}
\label{sec:org3907902}
\subsection{Team Organization}
\label{sec:org13e081e}
The three main components for this milestone are the symbol table,
type checking rules, and new AST; as well as tests for all
three. Development of these components was lead by Julian, David,
and Allan respectively. As there is a high degree of coupling
between each component, we continually sought feedback from one
another. The component leads are in charge of understanding the
overall component and in resolving concerns or requests from other
members.
\subsection{Contributions}
\label{sec:orgdbc7af8}
\begin{itemize}
\item \textbf{Julian Lore:} Implemented weeding of blank identifiers, symbol
table generation, typecheck (aside from type inference and
expression typechecking) and submitted invalid programs.
\item \textbf{David Lougheed:} Worked on expression type-checking and type inference,
including tests. Also worked on the weeding pass for return
statements.
\item \textbf{Allan Wang:} Added data structures for error messages, and supported
explicit error checking in tests. Created the data model for
symbol table core. Added hspec tests.
\end{itemize}
\end{document}