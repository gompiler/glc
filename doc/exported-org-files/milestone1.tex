% Created 2019-02-24 Sun 00:32
% Intended LaTeX compiler: pdflatex
\documentclass[11pt]{article}
\usepackage[utf8]{inputenc}
\usepackage[T1]{fontenc}
\usepackage{graphicx}
\usepackage{grffile}
\usepackage{longtable}
\usepackage{wrapfig}
\usepackage{rotating}
\usepackage[normalem]{ulem}
\usepackage{amsmath}
\usepackage{textcomp}
\usepackage{amssymb}
\usepackage{capt-of}
\usepackage{hyperref}
\date{\today}
\title{Design Document for Milestone 1}
\hypersetup{
 pdfauthor={},
 pdftitle={Design Document for Milestone 1},
 pdfkeywords={},
 pdfsubject={},
 pdfcreator={Emacs 26.1 (Org mode 9.2.1)}, 
 pdflang={English}}
\begin{document}

\maketitle
\tableofcontents

This document is for explaining the design decisions we had to make
whilst implementing the components for milestone 1.
\section{Scanner}
\label{sec:orgb78fc2f}
\subsection{One pass scanning for semicolon insertion}
\label{sec:orgc62b700}
Because our scanner is encompassed completely by \texttt{alexMonadScan} and
is interwoven with our parser with the \texttt{lexer} function in
\texttt{Scanner.hs} (all information is kept in a state monad and the
parser can request each token one at a time and do things lazily),
it was not feasible to do a second pass of the resulting tokens as
we don't get them in list form before passing them to the
parser. Our chosen solution was to use start codes in Alex and have
special rules for certain states, i.e. if the last token we scanned
was something that can take an optional semicolon, we'd enter the
\(nl\) state and encountering a newline in said state would scan the
newline to a semicolon, otherwise scanning anything else (that isn't
whitespace/ignored), would return to the default state (\(0\)), where
newlines are just ignored. This seems to be a nice/elegant solution
as we don't have to traverse the whole list or get any context of
any sort, other than the start code which is a feature built in to
Alex.
\subsection{Block comment support}
\label{sec:org1e423c7}
Alex seems to parse regexes one line at a time. The rule \texttt{/* .* */}
did not work for multiline block comments, so \texttt{checkBlk} was
implemented to iterate through the scanner's input and ignore
everything until we close the block comment. Additionally, this made
the task of outputting an error on unclosed blocks much easier (as
otherwise entering this new comment state would just run off to EOF
and not emit an error).

In addition, we had to account for semicolon insertion with block
comments, which we were able to do by adding a new case in \texttt{checkBlk}
that would set a semicolon flag to true if it encountered any new line
in the characters inside the block comment and then we were able to
insert a semicolon if the start code was \(nl\), which was conveniently
available for us.
\section{Parser}
\label{sec:orgba0bb92}
\subsection{AST decisions}
\label{sec:org4787625}
The AST is largely a one to one mapping of the Golang specs, with
parts we don't support removed and additional parts for Golite added.

In some cases, there are minor deviations from the CFG.

\begin{itemize}
\item We model our ast as accurately as possible, such that impossible
states are forbidden. We lack any checks for compatible types at
this stage, but we can match the definition for 'exactly one', 'one
or more', and 'zero or one'. In cases like identifiers, a \href{https://golang.org/ref/spec\#IdentifierList}{list} is
one or more (haskell \texttt{NonEmpty}), yet many locations make it
optional. While a direct translation would be \texttt{Maybe (NonEmpty a)},
we choose to make it \texttt{[a]} as it makes more sense.
\item Some splits, such as \texttt{add\_op} and \texttt{mul\_op} are distinguished purely
to demonstrate precedence. They are in fact only used once, so we
decide to merge them directly in our \texttt{ArithmOp} model. Several other
instances exist.
\item Given we created an AST vs a CST, we can further compact parts of
the grammar. For instance, and if clause in the spec leads to an
\texttt{IfStmt} grammar, whose \texttt{else} body is either a block (with
surrounding braces) or another if statement (no surrounding
braces). However, in our case, we don't need to model the braces, so
we can treat the else body exclusively as \texttt{Stmt} vs \texttt{Either Block
  IfStmt}
\item By design, our types for \texttt{int} and \texttt{string} specify whether they are
hex/octal/dec or raw/interpreted respectively. We kept this
information so that our pretty print would accurately represent the
input, even though we can convert them all to a single type (eg dec
and interpreted)
\item For var and type declaration, we make no distinction between single
declaration (exactly one) and block declaration (0 or more). Unlike
types, which produce different formats, we decide to enforce all
declarations of one var to be single declaration. In other words,
\texttt{var ( a = 2 )} would become \texttt{var a = 2}. Note that we cannot
further simplify group declarations \texttt{var ( a, b = 2, 3)}, as there
is no guarantee at this stage that the number of identifiers matches
the number of values. This would have to be checked at a later stage
\end{itemize}
\end{document}