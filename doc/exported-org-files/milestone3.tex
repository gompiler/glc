% Created 2019-03-27 Wed 23:18
% Intended LaTeX compiler: pdflatex
\documentclass[11pt]{article}
\usepackage[utf8]{inputenc}
\usepackage[T1]{fontenc}
\usepackage{graphicx}
\usepackage{grffile}
\usepackage{longtable}
\usepackage{wrapfig}
\usepackage{rotating}
\usepackage[normalem]{ulem}
\usepackage{amsmath}
\usepackage{textcomp}
\usepackage{amssymb}
\usepackage{capt-of}
\usepackage{hyperref}
\usepackage[margin=0.9in]{geometry}
\usepackage[fontsize=10.5pt]{scrextend}
\usepackage{enumitem}
\author{Lore, J., Lougheed D., Wang A.}
\date{\today}
\title{Design Document for Milestone 3}
\hypersetup{
 pdfauthor={Lore, J., Lougheed D., Wang A.},
 pdftitle={Design Document for Milestone 3},
 pdfkeywords={},
 pdfsubject={},
 pdfcreator={Emacs 26.1 (Org mode 9.2.2)},
 pdflang={English}}
\begin{document}

\maketitle
\tableofcontents

This document is for explaining the design decisions we had to make
whilst implementing the components for milestone 3.  \newpage
\section{JVM Bytecode for Code Generation}
\label{sec:orgc2daaec}
We decided on targeting JVM bytecode for our compiler, through the
\href{https://github.com/Storyyeller/Krakatau}{Krakatau}
bytecode assembler. Krakatau bytecode syntax is derived from Jasmin, but with
a more modern codebase (written in Python) and some additional features.
\subsection{Advantages}
\label{sec:org6449abb}
The primary advantages of targeting JVM bytecode are:
\hyperref[sec:org7cff5fc]{portability}, \hyperref[sec:orgd3b7f36]{execution speed},
and (surprisingly to us) its \hyperref[sec:org4cbab86]{focus on stack
operations, as opposed to a more `straightforward' language}, which
aids in overcoming some of the common pain points of GoLite code
generation:
\subsubsection{Portability}
\label{sec:org7cff5fc}
The JVM has been ported to many common platforms, meaning code written in
GoLite, when compiled with our compiler, will be able to run on any
platform the JVM can run on.
\subsubsection{Execution Speed}
\label{sec:orgd3b7f36}
Although Java is often considered slow as opposed to ahead-of-time compiled
languages such as C and C++ due to its garbage collection and non-native
compiled code, most implementations of the JVM provide JIT compilation.
By targeting JVM bytecode, we can take advantage of this, and our generated
code will likely be faster than if we generated code in a higher-level
language such as Python.

This JIT compilation and run-time optimization sometimes allow
the JVM to be faster than even ahead-of-time compiled programs, since
run-time information is available for optimization purposes.
\subsubsection{Stack Based/Low Level}
\label{sec:org4cbab86}
The fact that JVM bytecode is fairly low level gives us lots of granular
control for changing the behavior of constructs, especially when
dealing with odd \texttt{GoLite} / \texttt{GoLang} behavior that doesn't
map perfectly to more common programming languages.

Operating on a stack makes some of the operations mentioned in class as
`difficult to implement' surprisingly easy. In particular, swapping
variables (e.g. \texttt{a, b = b, a}) is fairly straightforward. The right-hand
side must be evaluated before assignment, which can be done by pushing and
evaluating the right-hand side, left to right, onto the stack; this is
followed by popping each value in turn and loading them into the
corresponding locals. Compared to (for example) temporary variable
allocation, this is an extremely natural way to implement this construct.
The stack also makes it easier to account for other constructs, such as
function arguments and comparisons. A similarly low-level, register-based
language would require the use of many temporary registers.

\subsection{Disadvantages}
\label{sec:org2fe95ff}
The main disadvantages of generating JVM bytecode are its low-level semantics
and its (in general) slightly slow speeds versus an ahead-of-time compiled
language.

The low level of JVM bytecode is particularly frustrating to deal with when
it comes to types that are not representable by an integer or floating point
number and are thus classes which must be instantiated; in the JVM, this
includes strings and \textasciitilde{}struct\textasciitilde{}s. This means that operations such as
comparisons and string concatenations go from being a few bytecode
instructions to signficantly longer patterns.
\section{Semantics}
\label{sec:org6cf3c86}
\subsection{Scoping Rules}
\label{sec:org7aa6ebe}
\subsubsection{Go Semantics}
\label{sec:orgf8cf82b}
In \texttt{GoLite}, new scopes are opened for block statements, \texttt{for}
loops, \texttt{if} / \texttt{else} statements and function declarations (for the
parameters and the function body). A new scope separates
identifiers (which are associated to type maps, variables,
functions and the constants true/false) from the other scopes'
identifiers.

Whenever we refer to an identifier, it will reference the
identifier declared in the closest scope.

There is nothing very special about scoping in \texttt{GoLite}, the main
notable thing is that something like \texttt{var a = a} will refer to \texttt{a}
in a previous scope, not the current \texttt{a} that was just declared,
unlike certain languages like \texttt{C}.
On the other hand, recursive types such as \texttt{type b b} fail as expected,
do not reference a type from higher scopes.
\subsubsection{Mapping Strategy}
\label{sec:orge137b79}
JVM bytecode, only has ``scoping'' for \texttt{methods}, as they have
their own locals and stack. However, block statements and the
statements inside of them do not have any scopes (except of course
method calls), as we do not have any constructs like loops, if
statements or switch statements. In a higher level language, we
could just append the scope to each identifier to keep them all
unique; renaming would also eliminate the need for separate scopes,
as we already typecheck the correct use of identifiers, and no further.
conflicts will arise. In our case, we use a similar strategy. Recall that
in the typecheck phase we generate a new checked AST with
simplified information and assumptions. The identifiers in this
AST also change, where they are tuples that contain the original
identifier and the scope they were declared in. Thus, each scoped
identifier refers to a unique declaration of a scoped identifier.

Given that our target language is JVM bytecode, our intermediate
representation will instead provide a unique index to each variable,
such that it remains one to one with the original scoped identifier key.
Variables with the same scoped identifier will be given the same offset,
and we will optimize our stack limit by reusing offsets when two variables
can never occur at the same time, due to branching.
\subsection{Switch Statements}
\label{sec:orge8e4d90}
\subsubsection{Go Semantics}
\label{sec:orgc959d88}
In \texttt{GoLite}, \texttt{switch} statements consist of an optional simple
statement, an optional expression and a (potentially empty) list
of case statements. Case statements are either a case with a
non-empty list of expressions, or a default case with no additional expression.
Each case statement also contains a block statement, containing code to execute
upon a match. This makes
them structurally different when compared to Java or \texttt{C} / \texttt{C++}, where:
\begin{itemize}[noitemsep]
\item Simple statements don't exist.
\item Expressions aren't optional.
\item Case statements don't match on a list of expressions.
\end{itemize}
The simple statement is executed before the \texttt{case} checking.
After that, the optional expression is compared with each \texttt{case}
statement, evaluating and comparing expression lists from left to
right. The first match enters that case's body, automatically
breaking at the end of it. This makes cases significantly different
semantically:
\begin{itemize}[noitemsep]
\item Cases automatically break.
\item Each \texttt{case} or \texttt{default} block defines its own scope for declarations.
\item Case statement expressions do not need to be a constant expression.
\end{itemize}

\subsubsection{Mapping Strategy}
\label{sec:org063df13}
For the structural differences:
\begin{itemize}[noitemsep]
\item Simple statements can be modeled and executed as the first statement in
the new ``scope''.
\item Missing expressions can be converted to the constant literal `true`.
\item For a list of expressions that is of length greater than one, we
can compare each element from the list one at a time, duplicating the
value we're comparing to before each comparison (as otherwise
we'll lose it during the stack operation). In other terms, the value we
compare to during each \texttt{case} block is stored on the stack until the
\texttt{switch} statement is done.
\end{itemize}
Semantically:
\begin{itemize}[noitemsep]
\item After \texttt{case} expression comparisons, we will either jump to the case body
or keep going to the next \texttt{case} comparison or \texttt{default} block.
\item To automatically break, for each case statement, we add a \texttt{goto}
to a label at the end of the switch statement.
\item \texttt{default} statement jumps should be placed after all jumps to case bodies
as the fall-through case, when no other jumps are followed.
\item Simulating new scopes is easy because of how our scoping works;
the variable names will already be resolved to their correct local's
index.
\item Expressions in \texttt{case} blocks not being constants does not matter too much
for us, as we will compare each expression normally (we are
simulating switch statements using \texttt{goto} and comparisons, and aren't
limited by any language-native \texttt{switch} statement definitions).
\end{itemize}
\subsection{Assignments}
\label{sec:org6c54d21}
\subsubsection{Go Semantics}
\label{sec:orgb465ec5}
In \texttt{GoLite}, assignments are either an assignment operator with a
single LHS expression and a RHS expression, or two non-empty
expression lists (LHS and RHS) of equal length. This makes them structurally
different (for the two non-empty list case) from `classic'
assignments, which typically only allow one l-value.
This structural difference is a lot more significant than it seems
at first glance, because the assignments are done in a ``simultaneous''
way, that is \texttt{a, b = b, a} will swap the values of \texttt{a} and \texttt{b}. If the
assignments were done sequentially, \texttt{a} and \texttt{b} would be the
original value of \texttt{b} and wouldn't be swapped. The same goes for \texttt{+=} and
other assignment operators.
\subsubsection{Mapping Strategy}
\label{sec:org486b55f}
There are two tricky things about assignments:
\begin{itemize}[noitemsep]
\item Assignment operators. We cannot just convert \texttt{e += e2} to \texttt{e =
      e + e2}, where \texttt{e} is an expression, because \texttt{e} might contain a
function call with side-effects, which we do not want to call
twice (note that in some cases, the assignment operator has an
equivalent bytecode instruction, i.e. incrementing and decrementing using
\texttt{iinc}. However, we generalize in this discussion as most
operators do not have an equivalent instruction to operate and
assign at the same time). There are thus several cases for \texttt{e}:
\end{itemize}
\begin{itemize}[noitemsep]
\item \texttt{e} is just an identifier. Then, we can just convert \texttt{e += e2}
to \texttt{e = e + e2}, as there will be no side effects.
\item \texttt{e} is a selector. If \texttt{e} is an addressable selector, then it
is not operating on the direct/anonymous return value of a
function call and so re-evaluating \texttt{e} will not produce any
side effects. Thus we can do \texttt{e = e + e2} again.
\item \texttt{e} is an index, say \texttt{e3[e4]}. In this case, \texttt{e3} can be an
anonymous \texttt{slice} from a function return, and \texttt{e4} could also be an
anonymous \texttt{int} from a function return. In order to avoid
duplicate side effects, we should resolve \texttt{e3[e4]}, including any
function calls, to some base addressable expression, storing the
result on the stack. Then, we can operate on the stack, adding \texttt{e2} and
assigning the result to whatever the stack value references.
\item The other cases for \texttt{e} are not \texttt{l-values}, and shouldn't happen
in the type-checked AST.
\end{itemize}
\begin{itemize}
\item Assignment of multiple expressions. As mentioned earlier, we
cannot do the assignments sequentially. Thus, we should evaluate the
entire RHS, pushing each result onto the stack and then
assigning each stack element one by one to their respective LHS
expression l-value. This way, \texttt{a, b = b, a} will not overwrite or
interfere with any values used on the RHS. This is one of the advantages
of using a stack-based language, as values on the stack implicitly act
like temporary variables, so we don't need to allocate other temporary
resources for simultaneous assignment.
\end{itemize}
\section{Currently Implemented: Intermediate Representation}
\label{sec:org67a14b2}
The main feature that was worked on during this milestone was the
creation of our intermediate representation, and the conversion of
the typechecked AST to said IR.

We decided on creating an IR for bytecode in order to make conversion easier
from the AST, and enforce some degree of correctness using Haskell's type
system. The IR is also stack-based, and to a large extent is functionally
identical to JVM bytecode, modeled in Haskell. We represent classes and
methods as Haskell records. Method bodies are a list of what we call,
\textasciitilde{}IRItem\textasciitilde{}s, which are either stack instructions or labels.

Available stack instructions, as of this milestone, include \texttt{Add} and other
binary operations, \texttt{Dup}, \texttt{Load} and \texttt{Store}, \texttt{InvokeVirtual/InvokeSpecial},
some integer-specific operations, and \texttt{Return\textasciitilde{}s. Instead of specifically
  representing equivalents of \textasciitilde{}iadd/fadd}, \texttt{iload/aload/...}, etc., we define
an \texttt{IRType} data type which can either be a bytecode primitive (integer or
float) or an object reference. In this way, the IR definition is kept short
and similar instructions can be combined into a single Haskell constructor
model. Other Haskell types are used to model method/class specifiers,
Jasmin-style parameter and return types, and loadable values (ints, floats,
and strings).

Eventually, our goal is to then convert this IR into Krakatau bytecode syntax,
which should be very straightforward given that the IR is so close to bytecode
already.
\end{document}