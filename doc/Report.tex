\documentclass[11pt]{article}
\usepackage[margin=1 in]{geometry}
\usepackage{hyperref, titling, fancyhdr, xcolor, lastpage, enumitem}

\author{\textsc{Lore}, J.\\ \textsc{Lougheed} D.\\ \textsc{Wang} A.}
\date{\today}
\title{Final Report}
\hypersetup{
  pdfauthor={Lore, J., Lougheed D., Wang A.},
  pdftitle={Final Report},
  pdfkeywords={Compiler Design} {GoLite} {Gompiler} {glc} {Final Report},
  pdfsubject={Final Report},
  pdflang={English},
  bookmarks=true,
  unicode=true,
  linktoc=all
}
\pagestyle{fancy}
\lhead{McGill University\\COMP 520: Compiler Design
  \\\thetitle}
\chead{\leftmark}
\rhead{\theauthor}
\cfoot{Page~\thepage~of~\pageref{LastPage}}

\newcommand{\todo}[0]{\textcolor{red}{\textbackslash\textbackslash TODO \ }}

\newcommand{\HRule}{\rule{\linewidth}{0.5mm}}

\begin{document}
\begin{titlepage}
  \center%

  \textsc{\LARGE McGill University}\\[1.5cm]
  \textsc{\Large COMP 520~-~Compiler Design}\\[0.5cm]
  \textsc{\large Gompiler~-~glc, a GoLite Compiler}\\[0.5cm]

  \HRule~\\[0.4cm]
  { \huge \bfseries \thetitle}\\[0.4cm]
  \HRule~\\[1.5cm]

  \theauthor\\%
  [3cm]

  {\large \thedate}\\[2cm]

  \vfill

\end{titlepage}
\tableofcontents
\newpage
\section{Introduction}
\subsection{Overview of Go}
% TODO
\subsection{Overview of GoLite}
% TODO
\subsection{Structure of Report}
In this report, we discuss the language and tool choices we made for
this project, as well as an overview of each compiler step (scanner,
parser, weeder, symbol table, typecheck and codegen respectively),
with major design decisions. Testing is also discussed.
\section{Language and Tool Choices}
\subsection{Implementation Language: Haskell}
% TODO MENTION CODEGEN TOOLS, i.e. Krakatau
For our \texttt{GoLite} compiler, we decided to use \texttt{Haskell}
as the implementation language, as many of the required tasks can be
done more naturally (i.e.\ tree traversal, recursing over self defined
structures, enforcing types, etc.).
\subsection{Tools}
For scanning, we use the program \texttt{Alex}\cite{github:alex}, a
lexical analyser generator which is similar to
\texttt{flex} but produces \texttt{Haskell} code instead. We use
\texttt{Happy}\cite{github:happy} for parsing, a parser generator which resembles
\texttt{bison/yacc}. \texttt{Alex} and \texttt{Happy} can be
interlinked nicely, passing on useful information such as the line number, column number, character
offset in file, and error messages.
\subsection{Miscellaneous}
We use Travis CI\cite{travisci} to build our project on every push to GitHub, as well
as run our tests to make sure that new changes did not break existing
functionality.

There are several other good Haskell scanning/parsing options, such as
\texttt{Megaparsec}. We decided on \texttt{Alex/Happy} because of
their syntactic similarity to \texttt{flex/bison}, which we learned in
class.

We use \texttt{stack}\cite{hs:stack} to manage our Haskell project and corresponding
tests.  We implemented additional tests using the \texttt{hspec}\cite{github:hspec}
package.

% TODO Do we need to mention file organization?
% \subsubsection{File Organization}
% \begin{itemize}
% \item \texttt{src/}
%   \begin{itemize}
%   \item \texttt{generated/} (Code generated by \texttt{golite.y},
%     using \texttt{Happy}, and \texttt{golite.x}, using
%     \texttt{Alex})
%   \item Modules for each feature
%   \end{itemize}
% \item \texttt{app/}
%   \begin{itemize}
%   \item \texttt{Main.hs} (main, imports the needed modules and uses
%     \texttt{ParseCLI} to parse the command line arguments and call
%     the function we want)
%   \end{itemize}
% \item \texttt{test/}
%   \begin{itemize}
%   \item \texttt{Spec.hs} (auto discovers \texttt{hspec} tests to run
%     them all with \texttt{stack test})
%   \item Spec modules, each exposing a test suite
%   \end{itemize}
% \end{itemize}

\subsection{Target Language: JVM Bytecode}
We decided on targeting JVM bytecode for our compiler, through the
Krakatau\cite{krakatau} bytecode
assembler. Krakatau bytecode syntax is derived from Jasmin, but with a
more modern codebase (written in Python) and some additional features.
\subsubsection{Advantages}
The primary advantages of targeting JVM bytecode are:
\hyperref[sec:portability]{portability},
\hyperref[sec:execs]{execution speed}, and its
\hyperref[sec:stack]{focus on stack manipulation, as opposed to a
  higher level language}, which aids in overcoming some of the common
pain points of GoLite code generation:
\paragraph{Portability}%
\label{sec:portability}
The JVM has been ported to many common platforms, meaning code written
in GoLite, when compiled with our compiler, will be able to run on any
platform the JVM can run on.
\paragraph{Execution Speed}%
\label{sec:execs}
Although Java is often considered slow as opposed to ahead-of-time
compiled languages such as C and C++ due to its garbage collection and
non-native compiled code, most implementations of the JVM provide JIT
compilation.  By targeting JVM bytecode, we can take advantage of
this, and our generated code will likely be faster than if we
generated code in an interpreted language such as Python. Sometimes,
the JVM can be faster than even ahead-of-time compiled programs, since
run-time information is available for optimization purposes.
\paragraph{Stack Based/Low Level}%
\label{sec:stack}
The fact that JVM bytecode is fairly low level gives us lots of
granular control for changing the behavior of constructs, especially
when dealing with odd \texttt{GoLite} / \texttt{GoLang} behavior that
doesn't map perfectly to more common programming languages.

Operating on a stack makes some of the operations mentioned in class
as `difficult to implement' surprisingly easy. In particular, swapping
variables (e.g. \texttt{a, b = b, a}) is fairly straightforward. The
right-hand side must be evaluated before assignment, which can be done
by pushing and evaluating all RHS terms, left to right, onto the
stack; followed by popping each value in turn and loading them into
corresponding locals. Compared to temporary variable or register
allocation, this is a very natural way to implement this construct.

\subsubsection{Disadvantages}
The main disadvantages of generating JVM bytecode are its low-level
semantics and its (in general) slightly slow speeds versus an
ahead-of-time compiled language.

The low level of JVM bytecode is particularly frustrating to deal with
when it comes to types that are not representable by an integer or
floating point number and are thus classes which must be instantiated;
in the JVM, this includes strings and \texttt{struct}. This means that operations
such as comparisons and string concatenations go from being a few
bytecode instructions to significantly longer patterns.
\section{Scanner}
\subsection{Overview}
% TODO ADAPT TO FINAL REPORT
The first step of this milestone was to scan user input into
tokens. Handing of most GoLite tokens was straightforward and similar
to class assignments. In other cases, we had to deal with some of
Go/GoLite's quirkiness when compared to something simple like
\texttt{MiniLang}.

\subsection{Design Decisions}
\subsubsection{Semicolon Insertion}
Semicolon insertion was the first scanning hurdle. To solve this, we
use start codes, a built in feature of \texttt{Alex}, and have special
rules for certain states.

If the last token we scan is something that can take an optional
semicolon, we enter the \(nl\) state, where newlines are transformed
into semicolons after specific tokens. Scanning anything else (that
isn't whitespace or otherwise ignored) returns the scanner to the
default state (\(0\)), where newlines are just ignored.

This solution seems elegant, as we don't have to traverse the whole
stream post-scanning or store context, other than the start code. It
also makes sense to let the scanner handle these situations
automatically, rather than trying to deal with them in the parsing
phase.
\subsubsection{Block comment support}
Block comments cannot be easily scanned using only regular
expressions. One potential solution was to use start codes again
(changing state upon opening a comment). This was ruled out since it
would potentially add a lot more start codes (i.e.\ states), as we'd
have combinatorial possibilities with the newline insertion state
(since block comments spanning newlines should also insert semicolons
when they're optional). Using a state approach would also make it
harder to catch unclosed comment block errors.

Our solution iterates through the scanner's input once we receive an
open comment block and ignore everything until the comment is closed.
If the comment never closes, we can easily emit an error.

To account for semicolon insertion with block comments, we set a
semicolon flag to true if we encounter any new line inside the block
comment, and we insert a semicolon if we're in the aforementioned
newline insertion state.
\paragraph{Adding newlines at the end of the file if they aren't
  present already}~\\
\texttt{Go} requires a semicolon or newline at the end of function
declarations. Sometimes, there may not be a newline at the end of a
file, so no semicolon insertion occurs, resulting in a parsing error.

In order to correct for this type of file, we enforce a final newline
by appending one if necessary. To do this, the code string is
preprocessed prior to scanning. This is the easiest solution, as
otherwise we would have to try inserting a semicolon if we are in the
\(nl\) state when we encounter an \texttt{EOF}, while also making sure
to return an \texttt{EOF}. This would be difficult without additional
context information.
\subsubsection{Nicer error messages}
We decided to use \texttt{ErrorBundle} from \texttt{Megaparsec}\cite{github:megaparsec} in
order to output nicer error messages, with program context for easier
debugging from an end-user programming perspective:

\begin{verbatim}
Error: parsing error, unexpected ) at 5:22:
  |
5 | func abstract(a, b, c) {
  |                      ^
\end{verbatim}

With \texttt{Alex}' default behaviour, we did not have access to the
entire source file string, as it is not kept between steps. In order
to generate the contextual message, we modified the \texttt{monad}
wrapper provided with \texttt{Alex} (see \texttt{TokensBase.hs}) and
changed the \texttt{Alex} monad to wrap over a \texttt{Either (String,
  Int) a} instead of \texttt{Either String a}, i.e.\ in addition to
storing an error message on the left side of the monad we also carry
an \texttt{Int} which represents the offset of the error. When we want
to print the error message, we can then append the part in the source
file where the error occurred.

\subsection{Testing}
\section{Parser}
\subsection{Overview}
% TODO POTENTIALLY ADD SECTION ABOUT PARSER ROLE LIKE IN SCANNER
\subsection{Design Decisions}
\subsubsection{Grammar}
% TODO ACCOUNT FOR COMMENTS IN M1 REPORT
Many of our difficulties in the grammar were associated with
identifier and expression lists, used in declarations/signatures and
assignment/function calls respectively. The grammar was refactored to
fix this by allowing identifier lists to become expression lists if
needed, in a way which avoided introducing other conflicts.

The first issue we encountered was with list ordering. LR parsers work
more intuitively with rules that put the newly-created terminal after
the recursively-expanding non-terminal. However, since Haskell uses
recursive lists defined in the opposite way, it is significantly more
efficient to prepend items. This prepending results in a reversed
ordering, which must be handled after the list is `complete'.

Adding an extra non terminal to manage reversals for each list would
needlessly increase our grammar and generated code size, so we decided
against it as a solution. The solution we use is to differentiate
lists containing at least one non-identifier expression (i.e.\ using
either all non-identifiers, identifiers plus one non-identifier, or
otherwise mixed lists, as grammar base cases) from lists of entirely
identifiers. Then, the expression list non-terminal is allowed to
yield either a mixed list or a pure identifier list depending on what
is needed.

% TODO ACCOUNT FOR M1 COMMENT "yikes"
Another caveat of how lists are handled in the grammar, again a
compromise to prevent ambiguity, is that the actual grammar constructs
that represent lists correspond to a list of size two or more, which
doesn't exactly match the Go spec (where a list may be 0/1 or more,
depending on the case). The actual single-item non-terminals are
allowed to represent a list of size one when needed, meaning this
disparity is resolved in the actual AST construct, which is closer to
a direct representation of the Go / GoLite specifications.
\subsubsection{AST}
% TODO ACCOUNT FOR M1 COMMENTS
We modeled our AST as close as possible to the actual Go and GoLite
specs, to try and ensure that impossible states are inherently
prevented by the Haskell type checker, reducing run-time errors.
Although we don't have type-checking implemented at this milestone, we
can use this technique to enforce definitions such as `exactly one',
`one or more', and `zero or one'. This modeling is not always
perfect. For example, a
list of
identifiers\cite{gospec:idlist} is `one or more' (in Haskell, \texttt{NonEmpty}). Many
locations make it optional. While a direct translation would be
\texttt{Maybe (NonEmpty a)}, we choose to make it a possibly empty
list \texttt{[a]} as it makes more sense.
\paragraph{Simplified Data Type Categories}
Some splits, such as \texttt{add\_op} and \texttt{mul\_op} are
distinguished purely to demonstrate precedence; they are in fact only
used once in the specs, so we decide to merge them directly in our
\texttt{ArithmOp} model. Several other instances exist.

Given we are creating an AST, rather than a CST, we can further
compact parts of the grammar. For instance, an \texttt{if} clause in
the spec leads to an \texttt{IfStmt} construction, whose \texttt{else}
body is either a block (with surrounding braces) or another
\texttt{if} statement (no surrounding braces). In our case, we don't
need to model the braces, so we can treat the \texttt{else} body
exclusively as \texttt{Stmt} rather than the more verbose
\texttt{Either Block IfStmt}. The grammar enforces that this
\texttt{Stmt} is not any other type.  Some splits, such as
\texttt{add\_op} and \texttt{mul\_op} are distinguished purely to
demonstrate precedence; they are in fact only used once in the specs,
so we decide to merge them directly in our \texttt{ArithmOp}
model. Several other instances exist.

Given we are creating an AST, rather than a CST, we can further
compact parts of the grammar. For instance, an \texttt{if} clause in
the spec leads to an \texttt{IfStmt} construction, whose \texttt{else}
body is either a block (with surrounding braces) or another
\texttt{if} statement (no surrounding braces). In our case, we don't
need to model the braces, so we can treat the \texttt{else} body
exclusively as \texttt{Stmt} rather than the more verbose
\texttt{Either Block IfStmt}. The grammar enforces that this
\texttt{Stmt} is not any other type.
\paragraph{Structure Simplification}
For \texttt{var} and \texttt{type} declaration, we make no distinction
between single declaration (exactly one) and block declaration (0 or
more). Unlike types, which produce different formats, we decide to
enforce all var declarations to be single declarations. In other
words, \texttt{var ( a = 2; b = 3 )} becomes \texttt{var a = 2; var b = 3}
and \texttt{var ( a = 2 )} would become \texttt{var a = 2}. Note
that we cannot further simplify multi declarations \texttt{var a, b
  = 2, 3} because they happen in one stage;
  \texttt{var a, b = b, a} does not always produce the same output as
  \texttt{var a = b; var b = a}.

\subsubsection{Pretty Printer}
When creating our pretty printer, we chose a top down approach.  Every
node has the ability to output a list of strings, which makes it
easier to format indentation. Each node is also only concerned with
its respective subtree, and does not require context from its
parent. We focused on aesthetics, focusing on proper spacing and
alignments. In the case of expressions, we tried to add brackets
sparingly, though further optimizations can be done down the road (a
nested binary op does not always need brackets, if the order of
precedence matches). To produce the full program, we simply join the
list of strings in the full program, intercalated with new lines.

\subsection{Testing}

To facilitate testing, we wanted to ensure that each feature can
be tested separately.
When building the parser, we expose entry points to each main stage,
so we can focus on tests for floats, expressions, statements, etc
on top of the full program.
To facilitate future refactoring, we base our tests on quality checks, vs structural checks.
In other words, we simply check if a parser succeeds ot fails.
This was very important, as testing the full ast structure meant that
it was very difficult to make changes in the future.

On top of the necessary tests with the provided scripts, we wanted as much
testing within Haskell as we can. Most inputs are embedded to avoid IO operations, and are done with the help of hspec so we can make use of Haskell's testing stack. This allows us to easily run all tests,
generate coverage reports, and rerun selective tests if we need to.

To ensure incremental updates, we wanted to make sure that we tested every step of the way. As a result, every build, with the exception of updating documentation, ran through Travis CI.

\todo DELETE
Each subsequent stage of the parser, eg typechecking, resource building, and codegen, also exposes a function for string inputs, so we can test them at various levels.
Stages relating to state, such as the symbol table and resource builder,
have their own high level module, where we can test event logic without
needing to parse a program. Together, this helps us develop incrementally,
and to a large extent independently from one another.

On top of the necessary tests with the provided scripts, we wanted as much
testing within Haskell as we can. Most inputs are embedded to avoid IO operations, and are done with the help of hspec so we can make use of Haskell's testing stack. This allows us to easily run all tests,
generate coverage reports, and rerun selective tests if we need to.

To ensure incremental updates, we wanted to make sure that we tested every step of the way. As a result, every build, with the exception of updating documentation, ran through Travis CI.
Towards later stages, we added docker support, since our test environment depended on many languages (Java, Haskell, Python).
In total, we have had nearly 1500 Travis builds, which helped us be confident with our changes.

\section{Weeder}
\subsection{Overview} %TODO
\subsection{Design Decisions}
% TODO M2 WEEDING
\subsubsection{Parser Weeding}
Most of the weeding operations needed are simple and don't rely on
external context. As a result, we were able to define recursive
traversal methods to verify relevant statements, and create verifiers
that validate at a single level. Haskell helped immensely here, as we
were able to use pattern matching to produce performant and
independent functions.

For verification where statement context was important (\texttt{break}
and \texttt{continue}) we made a simple modification to the recursive
traversal to avoid exploring scopes under \texttt{for} or
\texttt{switch} statements where applicable.

Each verifier returns an optional error, and we are able to map the
results and return the first error, if any.
\subsubsection{Typecheck Weeding}
We implemented additional weeding passes for certain constraints that
could be verified either at the weeding level or the typecheck level,
since in most cases it was easier to check via weeding. The
constraints we use weeding to check for this milestone are:
\begin{itemize}
\item Checking correct use of the blank identifier. It was much easier
  to recurse through the AST and gather all identifiers that cannot be
  blank, and then check this whole list, versus checking usage in the
  type-checking pass. Additionally, because we have offsets in our
  AST, we could easily point to the offending blank identifier without
  having to make error messages for each specific incorrect usage, as
  it is obvious what the incorrect usage is when we print out the
  location of the blank identifier.
\item Ensuring non-void function bodies end in return statements. It
  was easier to do a single weeding pass; otherwise, we'd have to
  recurse differently on functions that have a return type versus
  functions that don't have a return type during
  typechecking. Essentially, we'd need context-sensitive statement
  typechecking, with two different versions depending on return
  type. This was not deemed worth it compared to a single weeding
  pass.
  % TODO
  % ACCOUNT FOR M2 COMMENT, THIS HOLDS FOR ANY VOID FUNC
\item Ensuring \texttt{init} function declarations do not have any
  non-void return statements. Similar to the above, it is easier to do
  a weeding pass then require context-sensitive (i.e multiple
  different) traversal functions given the context of the current
  function declaration.
\item Ensuring any `main' or `init' function declarations do not have
  any parameters or a return type. This is again a straightforward
  weeding check.  It could also have been done via typecheck, but it
  is slightly easier to implement with weeding passes (no symbol table
  required).
\item Checking that any top-level declarations with the identifier
  \texttt{init} or \texttt{main} are functions, since they are special
  identifiers in Go (and GoLite) in the global scope.
\end{itemize}

\subsection{Testing} 

Testing weeding rules is relatively simple.
We simply provide a program that should parse, verify that it does parse,
and also verify that it fails the weeding rule.

At this stage, we decided to improve our testing process for invalid programs.
Given that error messages are unrelated to AST structure, we created a new error type class, where we represent error strings with data structures.
For instance, each weeding rule will have its own constructor, with optional arguments depending on how much detail we wish to add.
We then create a function that converts each data type to a string.

When testing for invalid programs, we can now also check that the message contains the string of our desired error.
Given that we have a unique prefix, we can ensure that there are no accidental matches with other errors.
This ended up being useful for two reasons: It allowed us to create errors in a more compact way, and avoid the possibility of making typos when pasting error message strings in numerous locations.
It also ensured that invalid programs were invalid for the reasons we expect.
After we refactored to the new model, we noticed that some of our old tests were indeed invalid for the wrong reasons.

\section{Symbol Table}
\subsection{Overview}
% TODO
\subsection{Design Decisions}
In Haskell, data structures are typically immutable, and much of the
language is designed around this. One of the main design decisions
made around the symbol table was deciding whether to go with an
immutable or mutable symbol table. In an immutable symbol table, a new
symbol table would have to be made every time a scope is added or
modified. Right away, despite this being a conceptually better fit for
the language, the potential performance degradation of constantly
re-building the symbol table becomes evident.

As a result of this performance impact, we decided on using a mutable
symbol table, with mutability supported via Haskell's
ST\cite{hackage:st}
monad. The constraint provides
runST\cite{hackage:runst}
(which removes the \texttt{ST}, i.e.\ mutability, from an interior
data structure). This is proven\cite{monadicstate} to keep functions pure. This made the \texttt{ST} monad a better choice than other
monads providing mutability (the main example being \texttt{IO}, which
if used would have resulted in all our functions after symbol table
generation being bound by \texttt{IO}, i.e.\ impure and also harder to
work with, as they are wrapped by an unnecessary monad). The trade-off
of this decision was a large increase in the difficulty of
implementing the symbol table, which made up a huge portion of the
work for this milestone.  However, once we finish using the symbol
table, our final result (a typechecked/simplified, proven-correct AST)
is pure and very easy to manipulate for the \texttt{codegen} phase in
the next milestone.

For symbol table storage, we created a new data type \texttt{Symbol}
to represent each symbol, analogous to the different type of symbols
(types, constants, functions, and variables). A \texttt{Symbol} can be
one of:
\begin{itemize}
\item \texttt{Base}: Base types (\texttt{int}, \texttt{float64}, etc.)
\item \texttt{Constant}: Constant values (only used for booleans in
  GoLite)
\item \texttt{Func [Param] (Maybe SType)}: Function types, with
  parameters and an optional return type.
\item \texttt{Variable SType}: Declared variables, of type SType.
\item \texttt{SType SType}: Declared types (note: in Haskell, the
  first \texttt{SType} here is a constructor, and the second
  \texttt{SType} is a data type attached to the constructor.)  In this
  way, all possible symbol types are encompassed by a single Haskell
  data type.
\end{itemize}

We also created a new data type \texttt{SType}, which is used to store
vital information concerning the types we define in the original
\texttt{AST}.  An \texttt{SType} can be one of:
\begin{itemize}
\item \texttt{Array Int SType}: An array with a length and a type.
\item \texttt{Slice SType}: A slice with a type.
\item \texttt{Struct [Field]}: A struct with a list of fields.
\item \texttt{TypeMap SIdent SType}: A user-defined type, with an
  identifier and an underlying type, which may be recursively mapped,
  eventually to a base type.
\item \texttt{PInt, PFloat64, PBool, PRune, PString}: Base types.
\item \texttt{Infer, Void}: Special
  \texttt{SType} for inferred types and void
  return values.
\end{itemize}
In this way, all possible GoLite types, including user-defined types,
are accounted for.

All types that are left to be inferred during symbol table generation
are updated when typechecking (we infer the type of variables and then
update their value in the symbol table, to make sure things like
assignments don't conflict with the original inferred type).
% TODO Do we need to mention scoping rules?
% \subsection{Scoping Rules}
% The scoping rules we used/considered are as follows:
% \begin{itemize}
% \item Function declarations: the parameters and function body are
%   put in a new scope, but the function itself is declared at the
%   current scope. Note that here we had to treat the function body as
%   a list of statements and not a block statement, because if we
%   recursed on the block statement our block statement rule would put
%   the function body in a new scope. Instead, the body must be in the
%   same scope as the parameters.
% \item Block statements are put into a new scope.
% \item If statements: we open a new scope, containing the simple
%   statement and expression condition at the top level, and then
%   other scope(s) inside for the body/bodies: one for \texttt{if},
%   and one for \texttt{else} (if there is one). If an \texttt{else}
%   is present, the if and else scope are siblings.
% \item Switch statements: open a new scope for the \texttt{switch},
%   and another scope for each switch \texttt{case}. All switch case
%   scopes are siblings.
% \item For loop: open a new scope, with optional clauses put at the
%   top level (simple statements 1 and 2, and condition). The body is
%   put in a nested scope.
% \end{itemize}

\subsection{Testing} %TODO
% TODO ACCOUNT FOR M2 COMMENT Try to describe some of the type rules
% you use for typecheck specific stmts/exprs if they are interesting
\section{Typecheck}
\subsection{Overview}
% TODO
\subsection{Design Decisions}
For type-checking, we decided on a single-pass approach, combining
symbol table generation and statement type-checking. This improves
performance, and is feasible as a product of GoLite's declaration
rules, which specify that identifiers must be declared before they can
be used.

The other approach we considered involved generation of a
type-annotated AST, with types of expressions contained in the AST, so
that we could get rid of \texttt{ST} mutability from the symbol table
as soon as possible (some of us did a similar thing for the
assignment, but this was mainly relevant for print statements in C
codegen needing to know the type of the expression they're printing).

We decided on doing all typechecking at the same time as symbol table
generation because type inference has to be done to generate this new
AST, and type inference requires typechecking (e.g. \texttt{"a" + 5}
has no inferred type, since the expression is undefined; we only know
this because of typechecking).

At first, we were going to generate an annotated AST only to typecheck
things that aren't expressions. However, at that point, since we were
already doing one in-depth pass of the original AST for symbol table
generation, we decided that we might as well do the other half of
typechecking in the same phase, since it seemed odd to split
typechecking between the symbol table and a separate pass. The
alternate approach may have been more feasible if type inference did
not require typechecking, but in GoLite it did not seem to make
sense. Therefore, after the one pass of our original AST, the final
result is a typechecked AST, with extremely limited type annotation.

Additionally, we decided to resolve all type mappings (except for
structs) to their base types when generating this new AST:~all the
casts/equality checks/new type usages are already validated in
typechecking, so we don't need them anymore, nor do we need the
mappings. Thus, our new AST was also able to get rid of type
declarations (except for structs).
\subsubsection{New AST}
As mentioned above, dependency on the SymbolTable results in a
dependency on the \texttt{ST} monad, which adds complexity to each
operation.  As a result, our goal after typechecking is to create a
new AST, which reflects the new constraints we enforce.  Namely:
\begin{itemize}
\item Typecheck errors are caught beforehand, so we no longer need
  offsets, or error breakpoints.
\item All variables are properly typechecked, and can therefore
  reference an explicit type. Each type is composed of parent types up
  until the primitives.  This includes cases like function signatures,
  where we can associate each parameter with a type instead of
  allowing lists of identifiers to map to a single type.  In
  preparation for codegen, we can then use our new AST exclusively,
  without any other mutable data structures. Any additional
  information we need can be added back into the AST, with minimal
  changes to models used at previous stages.
\end{itemize}

An added benefit of adding items to the AST is to support caching.
Haskell computes data lazily, so having any argument within a 
data type will ensure at most one computation.
As shown above, providing a way of getting expression types is very important. While we do have all the information we need within the previous AST,
we felt it was inefficient to require recursion through binary and unary expressions.
Additionally, recursion for binary expressions may lead to errors in the 
event that the two types do not match. To avoid this possibility,
we simply offload the verification to the symbol table, and assume that the provided type is valid.
We add the explicit type to the ast, and now future stages simply require 
a field access to find it.

\subsection{Testing} %TODO
% TODO Do we need to mention our invalid programs?  Summary of the
% check in each invalid program:
% \begin{itemize}
% \item \texttt{append-diff-type.go}: Append an expression of a
%   different type than the type of the expressions of the
%   \texttt{slice}.
% \item \texttt{append-no-slice.go}: Append to something that isn't a
%   slice.
% \item \texttt{assign-no-decl.go}: Assign to a variable that hasn't
%   been declared.
% \item \texttt{assign-non-addressable.go}: Assign to a LHS that is a
%   non-addressable field.
% \item \texttt{cast-not-base.go}: Cast to a type that isn't a base
%   type.
% \item \texttt{dec-non-lval.go}: Decrement something that isn't an
%   \texttt{lvalue}.
% \item \texttt{decl-type-mismatch.go}: Declare and assign variable of
%   explicit type to an expression of a different type.
% \item \texttt{float-to-string.go}: Try to cast a \texttt{float} to a
%   \texttt{string}.
% \item \texttt{for-no-bool.go}: While variant of for loop with a
%   condition that isn't a bool.
% \item \texttt{func-call.go}: Function call with arguments of
%   different type than function declaration arguments.
% \item \texttt{func-no-decl.go}: Calling a function that hasn't been
%   declared.
% \item \texttt{function-already-declared.go}: Trying to declare a
%   function that has already been declared.
% \item \texttt{function-duplicate-param.go}: Trying to declare
%   function with two params with same name.
% \item \texttt{if-bad-init.go}: If with an init statement that does
%   not typecheck (assignment of different type).
% \item \texttt{inc-non-numeric.go}: Increment an expression that
%   doesn't resolve to a numeric base type.
% \item \texttt{index-not-list.go}: Index into something that isn't a
%   slice.
% \item \texttt{index.go}: Index that does not resolve to an int.
% \item \texttt{invalid-type-decl.go}: Declare a type mapping to a
%   type that doesn't exist.
% \item \texttt{no-field.go}: Using selector operator on struct that
%   doesn't have the field requested.
% \item \texttt{non-existent-assign.go}: Assigning a variable to a non
%   existent variable.
% \item \texttt{non-existent-decl.go}: Trying to declare a variable of
%   a type that doesn't exist.
% \item \texttt{op-assign.go}: Op-assignment where variable and
%   expression are not compatible with operator (i.e. \texttt{int +
%   string})
% \item \texttt{print-non-base.go}: Trying to print a non base type.
% \item \texttt{return-expr.go}: Returning an expression of different
%   type than the return type of the function.
% \item \texttt{return.go}: Return nothing from non-void function.
% \item \texttt{short-decl-all-decl.go}: Short declaration where all
%   variables on LHS are already declared.
% \item \texttt{short-decl-diff-type.go}: Short declaration where
%   already defined variables on LHS are not the same type as assigned
%   expression.
% \item \texttt{switch-diff-type.go}: Type of expression of case is
%   different from switch expression type.
% \item \texttt{type-already-declared.go}: Trying to define a type
%   mapping to a type that already exists.
% \item \texttt{var-already-declared.go}: Trying to declare a variable
%   that is already declared.
% \end{itemize}

\section{Code Generator}
\subsection{Overview}
% TODO
\subsubsection{Scoping in GoLite}
In \texttt{GoLite}, new scopes are opened for block statements,
\texttt{for} loops, \texttt{if} / \texttt{else} statements and
function declarations (for the parameters and the function body). A
new scope separates identifiers (which are associated with type maps,
variables, functions and the constants \texttt{true} and
\texttt{false}) from the other scopes' identifiers. Whenever we refer
to an identifier, it references the identifier declared in the closest
scope.

There is nothing very special about scoping in \texttt{GoLite}. The
main notable thing is that something like \texttt{var a = a} will
refer to \texttt{a} in a previous scope, not the current \texttt{a}
that was just declared, unlike languages like \texttt{C}.  On the
other hand, recursive types such as \texttt{type b b} fail as
expected, and do not reference a type from higher scopes.
\subsubsection{Switch Statements in GoLite}
In \texttt{GoLite}, \texttt{switch} statements consist of an optional
simple statement, an optional expression and a list of case
statements. Case statements are either a case with a non-empty list of
expressions, or a default case with no additional expression.  Each
case statement also contains a block statement, containing code to
execute upon match. This makes them structurally different when
compared to Java or \texttt{C} / \texttt{C++}, where:
\begin{itemize}[noitemsep]
\item Simple statements don't exist.
\item Expressions aren't optional.
\item Case statements don't match on a list of expressions.
\end{itemize}
The simple statement is executed before the \texttt{case} checking.
After that, the optional expression is compared with each
\texttt{case} statement, evaluating and comparing expression lists
from left to right. The first match enters that case's body,
automatically breaking at the end of it. This makes cases
significantly different semantically:
\begin{itemize}[noitemsep]
\item Cases automatically break.
\item Each \texttt{case} or \texttt{default} block defines its own
  scope for declarations.
\item Case statement expressions do not need to be a constant
  expression.
\end{itemize}
% TODO ACCOUNT FOR M3 COMMENT IN THIS SEC AND MAPPING STRATEGIES What
% about assigning arrays and structs ? What about assigning slices ?
% There are no mentions of the copying mechanisms.
\subsubsection{Assignments in GoLite}
In \texttt{GoLite}, assignments are either an assignment operator with
a single LHS expression and a RHS expression, or two non-empty
expression lists (LHS and RHS) of equal length. This makes them
structurally different (for the two non-empty list case) from
`classic' assignments, which typically only allow one l-value.  This
structural difference is a lot more significant than it seems at first
glance, because the assignments are done in a ``simultaneous'' way,
that is \texttt{a, b = b, a} will swap the values of \texttt{a} and
\texttt{b}. If the assignments were done sequentially, \texttt{a} and
\texttt{b} would be the original value of \texttt{b} and wouldn't be
swapped.

\subsection{Mapping Strategies}
\subsubsection{Scoping}
JVM bytecode only has ``scoping'' for \texttt{methods}, as they have
their own locals and stack. Block statements do not exist per se, and
statements except called method bodies are scope-less. In higher level
languages, we could just append the scope depth to all identifier
names to keep them unique, also eliminating the need for separate
scopes, as we already typecheck the correct use of identifiers. In our
case, we have scoped identifiers in our newly generated typechecked
AST which we convert to offsets (for locals).

These offsets can be used in our intermediate representation, where
the offsets are unique for each variable declaration. Variables with
the same scoped identifier will be given the same offset, and we can
optimize our stack limit by reusing offsets when two variables can
never occur at the same time due to branching.
\subsubsection{Switch Statements}
For the structural differences:
\begin{itemize}[noitemsep]
\item Simple statements can be modeled and executed as the first
  statement in the new ``scope''.
\item Missing expressions can be converted to the constant literal
  `true`.
\item For a list of expressions that is of length greater than one, we
  can compare each element from the list one at a time, duplicating
  the value we're comparing to before each comparison (as otherwise
  we'll lose it during the stack operation). In other terms, the value
  we compare to during each \texttt{case} block is stored on the stack
  until the \texttt{switch} statement is done.
\end{itemize}

\noindent Semantically:

\begin{itemize}[noitemsep]
\item After \texttt{case} expression comparisons, we will either jump
  to the case body or keep going to the next \texttt{case} comparison
  or \texttt{default} block.
\item To automatically break, for each case statement, we add a
  \texttt{goto} to a label at the end of the switch statement.
\item \texttt{default} statement jumps should be placed after all
  jumps to case bodies as the fall-through case, when no other jumps
  are followed.
\item Simulating new scopes is easy because of how our scoping works;
  the variable names will already be resolved to their correct local's
  index.
\item Expressions in \texttt{case} blocks not being constants does not
  matter too much for us, as we will compare each expression normally
  (we are simulating switch statements using \texttt{goto} and
  comparisons, and aren't limited by any language-native
  \texttt{switch} statement definitions).
\end{itemize}
\subsubsection{Assignments}
There are two tricky things about assignments:
\begin{itemize}[noitemsep]
\item Assignment operators. We cannot just convert \texttt{e += e2} to
  \texttt{e = e + e2}, where \texttt{e} is an expression, because
  \texttt{e} might contain a function call with side-effects, which we
  do not want to call twice (note that in some cases, the assignment
  operator has an equivalent bytecode instruction, i.e.\ incrementing
  and decrementing using \texttt{iinc}. However, we generalize in this
  discussion as most operators do not have an equivalent instruction
  to operate and assign at the same time). There are thus several
  cases for \texttt{e}:
  \begin{itemize}[noitemsep]
  \item \texttt{e} is just an identifier. Then, we can just convert
    \texttt{e += e2} to \texttt{e = e + e2}, as there will be no side
    effects.
  \item \texttt{e} is a selector. If \texttt{e} is an addressable
    selector, then it is not operating on the direct/anonymous return
    value of a function call and so re-evaluating \texttt{e} will not
    produce any side effects. Thus we can do \texttt{e = e + e2}
    again.
  \item \texttt{e} is an index, say \texttt{e3[e4]}. In this case,
    \texttt{e3} can be an anonymous \texttt{slice} from a function
    return, and \texttt{e4} could also be an anonymous \texttt{int}
    from a function return. In order to avoid duplicate side effects,
    we should resolve \texttt{e3[e4]}, including any function calls,
    to some base addressable expression, storing the result on the
    stack. Then, we can operate on the stack, adding \texttt{e2} and
    assigning the result to whatever the stack value references.
  \item The other cases for \texttt{e} are not \texttt{l-values}, and
    shouldn't happen in the type-checked AST.\@
  \end{itemize}
  \begin{itemize}
  \item Assignment of multiple expressions. As mentioned earlier, we
    cannot do the assignments sequentially. Thus, we should evaluate
    the entire RHS, pushing each result onto the stack and then
    assigning each stack element one by one to their respective LHS
    expression l-value. This way, \texttt{a, b = b, a} will not
    overwrite or interfere with any values used on the RHS. This is
    one of the advantages of using a stack-based language, as values
    on the stack implicitly act like temporary variables, so we don't
    need to allocate other temporary resources for simultaneous
    assignment.
  \end{itemize}
\end{itemize}
\subsection{Testing} %TODO
\section{Conclusion}
\subsection{Experience} %TODO
\subsection{What We'd Change}

While Haskell has allowed us to easily define types, certain
restrictions still allow for invalid AST structures.  For instance, we
can only get the length of an array or a slice, yet the AST support
all expressions.  To properly restrict all types, we would need to
further group data into multiple subsets of constructors\cite{so:pattern/matching/subset}.  While this
would catch more problems at compile time, it will also make the code
significantly more complex.  The other approach is to allow errors at
all stages of the compiler, even though all valid typechecked code has
a code gen counterpart.  This adds multiple catches of the same error
at runtime, but keeps the code short.  Our current approach is to
`throw' errors when they occur, without providing a function signature
that indicates errors can occur.  This allows us to keep our code
short, but is problematic with evolution, as users would expect a pure
function to be pure (void of errors).

\section{Contributions}
% TODO COMBINE CONTRIBS TODO M3/M4 CONTRIBS M1 CONTRIBS BELOW
\begin{itemize}
	\item \textbf{Julian Lore}
	\begin{itemize}
		\item M1: Wrote the majority of the scanner and
		handled weird cases, wrote a large amount of valid/invalid programs,
		implemented many other tests (\texttt{hspec} or small tests in our
		program) and looked over the parser, contributing a few things to it
		as well.
		\item M2: Implemented weeding of blank identifiers,
		symbol table generation, typecheck (aside from type inference and
		expression typechecking) and submitted invalid programs.
		\item M3/M4: \todo
	\end{itemize}
	\item \textbf{David Lougheed}
	\begin{itemize}
		\item M1: Wrote the bulk of the parser grammar
		and contributed to the weeder. Also wrote 3 of the valid programs
		and 8 of the invalid ones and had minor contributions to
		miscellaneous other components.  Contributed to the testing of the
		parser and pretty printer.
		\item M2: Worked on expression type-checking and
		type inference, including tests. Also worked on the weeding pass for
		return statements.
		\item M3/M4: \todo
	\end{itemize}
	\item \textbf{Allan Wang}
	\begin{itemize}
		\item M1: Created the AST and helper classes for
		pretty printing and error handling.  Wrote the base package for
		testing as well as some of the embedded test cases within
		\texttt{hspec}.  Added integrations (Travis + Slack), and gave code
		reviews to the other components.
		\item M2: Added data structures for error messages,
		and supported explicit error checking in tests. Created the data
		model for symbol table core. Added hspec tests.
		\item M3/M4: Created resource AST, which replaces local identifiers with stack indices, collects struct definitions, and reorganizes init functions. Created GlcArray and struct template in Java. Contributed to tests.
	\end{itemize}
\end{itemize}

\newpage

\bibliographystyle{plain} \bibliography{ref}
\end{document}
